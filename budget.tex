\section{Budget}


\subsection{Revenue}
Registration costs for the NC State Student Conference will be \$40 per student and \$250 per professional. With an anticipated attendance, we are expecting \$36,750. Sponsorship funds are estimated at \$182,250. This number does not take into account money typically provided by ANS National Divisions & Regional Sections. The ANS National provides seed monies that are to be reimbursed.

Registration costs at the VCU conference will be \$35 per student and \$250 per professional. With our anticipated attendance, we are expecting \$40,750. We are estimating \$187,500 in sponsorship funds bringing our total revenue to \$228,250. This number does not take into account the money typically provided by ANS Sections and through the ANS National Seed money. We believe with our proximity to so many large nuclear companies, this sponsorship will be achieved.

Complete sponsorship data from the 2014 and 2016 conferences was provided by the Penn State and Wisconsin Conference Co-Chairs. Overall sponsorship figures from 2015 were provided by the Texas A&M conference leadership. Each set of data was carefully analyzed by our sponsorship team in order to develop a reasonable sponsorship projection. As shown, we expect a total of 56 waived professional registrations. This correlates to the number of waived fees given in our expected tier sponsorship figures and also factors in discounts to speakers, panelists, and workshop instructors.

Taking into account the revenues of past student conferences, we have estimated our sponsorship revenue to amount to \$175,500. Additionally, we have chosen to set our registration fees at \$35 for students and \$250 for professionals. Therefore, we expect \$53,250 from registration. In total, our revenue is expected to be \$228,750 which we believe is reasonable and conservative in light of past revenues. In addition, our estimates provide a difference with our expected expenditures. In this way, we are less likely to have to cut things from the budget or raise registration fees if our sponsorship goals are not met. -By setting the highest sponsorship level to be \$20,000 which is lower than in the past, we hope to attract three top sponsors to name the three conference dinners. This causes our overall revenue to be at a reasonable medium between Penn State and Texas A&M’s. -Lunch & learns have become very popular. Through advice from past chairs, we are requiring at least a \$10,000 contribution to host a lunch & learn so that we might attract more funding. Therefore, we anticipate that this might encourage otherwise \$5,000 or \$7,500 level sponsors to be \$10,000 sponsors. We hope to have more \$10,000 sponsors but then perhaps fewer \$5,000 sponsors. -ANS Divisions tend to donate \$1,000 to \$2,500.

Matching this year’s conference at Texas A&M, early member registration fees at the UW-Madison conference will be \$35 per student and \$250 per professional. With a projected attendance of 500 students and 150 professionals, revenue from registration is projected to total \$55,000. 

\subsection{Expenses}

\subsection{Sponsorship Plan}
The full cost of the conference will be covered through sponsorships from various entities. We will first reach out to local nuclear industries in North Carolina such as GE Hitachi in Wilmington (headquarters of their nuclear division), Duke Energy in Charlotte (headquarters) and Westinghouse Electric Company, to name a few. Next, companies that helped sponsor past ANS Conferences would be contacted in hopes of acquiring their continued support. And finally, we will seek sponsorship from the ANS itself and the NC State Engineering Department. To encourage large donations, we have set up different sponsorship packages.

We plan to offer a variety of different sponsorship options that provide several levels of conference visibility. Through these incentives, sponsoring companies and organizations will be exposed to arguably the highest concentration of young talent available in the nuclear field. Our highest tier sponsors will get the most conference visibility by selecting one of our Elite Packages. These packages include the \$5000 Ally Package, the \$10,000 Advocate Package, the \$15,000 Leader Package, and lastly, the \$20,000 Champion Package. Incentives for each are outlined on the following page. In addition to Elite Packages, we’re offering three separate exhibitor packages. In this approach, we are offering a little something extra at each level. For those who are only interested in participating in the Career Fair, we have our Career Fair Basic package. For only \$1000 dollars more, sponsors will be provided additional incentives on top of hosting a career fair table. Lastly, we offer the Exhibitor Premier packages. With a limited availability, the Exhibitor Premier package offers the large additional incentive of hosting either the Friday or Saturday Lunch & Learns. Sponsors selecting this package will be interested to know that we are planning these Lunch & Learns in such a way we feel will maximize the number of attendees. This is done by creating little overlap between other conference activities and creating dining arrangements that make it most convenient for attendees to enter the Grand Ball Room where the Lunch & Learns are set to take place. For those who would like to contribute at any level below the exhibitor package levels, we have our Friends of Nuclear Contributor package. Any ANS Division who is a Friends of Nuclear Contributor will be offered two tables at the Divisions of ANS Policy Dinner.

Upon the announcement that the 2016 ANS student conference is to be held in Madison, we will look towards meeting our revenue expectations through fundraising. Our first action will be to contact industry partners who have consistently supported past ANS conferences and our own ANS student section. Letters will be made detailing the conference’s plan and the great opportunity they have to get involved. Financial contributions will receive different incentives based on donation amount, detailed in Section 5.2.3. Two sources of support we already anticipate are from both the UW College of Engineering and the Engineering Physics department. We have spoken with one of the College of Engineering Deans and the Department Chair about our vision for the conference and their role in making this a reality. Beyond their written support (shown in Appendix G) they each are willing to sponsor at least \$2,000 towards the conference. In addition, the EP department is willing to help us reach out to our extensive alumni network for financial support of this conference. Conversations with previous conference hosts indicated the ANS divisions as another consistent source of support. Individual divisions give smaller donations, but as a whole they have consistently given \$36,000 in total to student conferences in recent years. As such, their contributions will also be recognized in our program and in lists of conference sponsors. In the past, local businesses have provided financial or other types of support for ANS student conferences. We will attempt to continue this trend and contact restaurants, for donations, either monetary or in the form of conference discounts. Due to the uncertain nature of these contributions, we have chosen to be safe and not include them in our revenue estimates.

\subsection{Banking}

In order to properly handle all of the conference expenses, two accounts will be used. One account will be with the national ANS headquarters, and the second will be a local checking account through Busey Bank. Our student section currently holds a checking account with Busey Bank and has developed a working relationship with them. Similar to previous conferences, the national ANS account will be the primary account due to their previous experience in handling conference funds and their 501(c)(3) tax exemption status. Although an account with the national ANS organization is currently not maintained by the UIUC local chapter, should this bid be selected, the opening of an account would happen almost immediately. An established local checking account with Busey Bank allows for convenience when using them as a secondary account. The student section account has been maintained for a number of years, allowing for adequate knowledge of Busey’s policies and a stable relationship to be established with the local branch. A new account would be opened, such that the student chapter funds and the conference funds are completely separate and require different oversight. This account will be used for small expenses that can occur during the conference. Although the ANS-managed account could be used for such purposes, the presence of a local branch allows for more flexibility if a purchase becomes time sensitive. Should the additional Busey Bank account be unobtainable, an account would be established with Chase instead. If ANS National desires to manage all funds, accommodations will be made to consolidate the funds.

\subsubsection{Financial Oversight}
Financial integrity must be maintained when providing a conference of this size. To do so, diligent oversight will be practiced for all transactions related to the conference. Expense requests will be required for all transactions. These requests must carefully outline the reason for the purchase and the total cost. If the purchase is reoccuring, automation will be required at the time of request. These requests will require the approval of both conference chairs in addition to the financial director. Only the conference chairs and the financial director will have authority, assuming the previously mentioned permission, to complete transactions on the Busey Bank account. To ensure transparency, the financial director will update a public record containing all transactions. 

\subsection{Financial Contingency}

\subsection{Cost of Attendance and Student Reimbursement}
The cost of attending the student conference, per student, varies among schools and is highly dependent on distance from UIUC and the preferred mode of travel. We will assume that minimizing cost is a priority for schools, thus number of persons per hotel room is double the number of beds. 