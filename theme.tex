\section{Saving the World One Atom at a Time}
The future is nuclear.\\
There are many challenges facing the world today and some have been designated existential threats to humanity. Young people today will witness the growing toll of anthropogenic climate change. As students, obstacles at the scale of the world climate crisis appear daunting and overwhelming. We believe that many solutions will come from the nuclear sciences. The ANS Student Conference is an opportunity for students and professionals to come together and share advances in critical technology and research, dedicated to solving these problems. Nuclear, plasma, and radiological engineering will be central to many endeavors. Whether the goal is solving the world’s energy needs, developing technology that will take us to the stars, or curing cancer. By hosting this conference, we hope to inspire and motivate students in nuclear, plasma, and radiological engineering fields to tackle big problems. Saving the World One Atom at a Time reflects the fact that nuclear science is a powerful force in dealing with grand challenge problems. This theme also honors the individual, atomic, contributions from students, researchers, and professionals that are essential to progress. This conference is about science and it is about the people that make science possible. Students will hear from visionary speakers and leaders of the nuclear science community and come away with optimism for the future; knowing that they are saving the world one atom at a time.\\
The University of Illinois at Urbana-Champaign chapter of ANS would be honored to host the 2021 student conference. We hope to create an atmosphere that will galvanize students and professionals for the exciting future of nuclear engineering.\\

Goals of the conference
\begin{enumerate}
	\item Celebrate the people behind the science
	\item Inspire young students to take on grand challenge problems
	\item Help students and professionals develop a strong network of like-minded people. 
\end{enumerate}