\section{Saving the World One Atom at a Time}
The future is nuclear. There are many grand challenges facing the world today and some have been designated existential threats to humanity. Young people today will witness the growing toll of anthropogenic climate change. As students, obstacles at the scale of the world climate crisis appear daunting and overwhelming. We believe that many solutions will come from the nuclear sciences. The ANS Student Conference is an opportunity for students and professionals to come together and share advances in critical technology and research dedicated to solving these problems. Nuclear, plasma, and radiological engineering will be central to many endeavors, whether the goal is solving the world’s energy needs, developing technology that will take us to the stars, or curing cancer. By hosting this conference, we hope to inspire and motivate students in these engineering fields to tackle big problems. Saving the World One Atom at a Time reflects the fact that nuclear science is a powerful force in dealing with grand challenge problems. This theme also honors the individual, atomic, contributions from students, researchers, and professionals that are essential to progress. This conference is about science and engineering and it is about the people that make science and engineering possible. Students will hear from visionary speakers and leaders of the nuclear science community and come away with optimism for the future; knowing that they are saving the world one atom at a time.\\
There are three main goals of our theme and each of these goals will be the focus of a different day of the student conference.
\begin{enumerate}
	\item Celebrate the people behind the science and engineering.\\
	Everyone that does science has a unique background, skillset, and experiences. People are what make science possible. Encouraging diversity and inclusivity in these areas, and others, improve creativity, productivity, and insights. We showcase this aspect of the conference with panels like \textit{Science is People: How to Conduct Inclusive Research} and \textit{Scientific Storytelling}.
	\item Connect students and professionals to develop strong networks.\\
	This is the networking and professional-development-focused section of the conference. Beyond the mere job-seeking aspect, these networks enable the spread of ideas. Additionally, this evening builds on top of the previous night as we remember that our networks also consist of people. More diverse networks are better networks. This aspect of the conference is captured by workshops like \textit{Developing Your Network}. 
	\item Inspire the next generation of nuclear engineers to take on grand challenge problems.\\
	The final day is the culmination of the conference and underpins the single unifying ideal. For many students, this conference might be the first time they are presenting research to their peers, mentors, and future employers. Everything about this conference should encourage students to take on challenges that seem bigger than they are in order to improve the world around them.  
\end{enumerate}
These goals and our theme motivated every decision in our conference proposal. The University of Illinois at Urbana-Champaign chapter of ANS would be honored to host the 2021 student conference. We hope to create an atmosphere that will galvanize students and professionals for the exciting future of nuclear engineering.\\
